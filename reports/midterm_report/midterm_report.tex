\documentclass[12pt, english, openany]{book}

\usepackage[]{graphicx}
\usepackage[]{color}
\usepackage{alltt}
\usepackage[T1]{fontenc}
\usepackage[utf8]{inputenc}
\setcounter{secnumdepth}{3}
\setcounter{tocdepth}{3}
\setlength{\parskip}{\smallskipamount}
\setlength{\parindent}{0pt}

\usepackage[top=100pt,bottom=100pt,left=68pt,right=66pt]{geometry}

\raggedbottom

\usepackage[english]{babel}

% All page numbers positioned at the bottom of the page
\usepackage{fancyhdr}
\fancyhf{} % clear all header and footers
\fancyfoot[C]{\thepage}
\renewcommand{\headrulewidth}{0pt} % remove the header rule
\pagestyle{fancy}

% Changes the style of chapter headings
\usepackage{titlesec}
\titleformat{\chapter}
   {\normalfont\LARGE\bfseries}{\thechapter.}{1em}{}
% Change distance between chapter header and text
\titlespacing{\chapter}{0pt}{50pt}{2\baselineskip}

% Adds table captions above the table per default
\usepackage{float}
\floatstyle{plaintop}
\restylefloat{table}

% Adds space between caption and table
\usepackage[tableposition=top]{caption}

% Adds hyperlinks to references and ToC
\usepackage{hyperref}
\hypersetup{hidelinks,linkcolor = blue} % Changes the link color and hides the
% hideous red border that usually is created

% If multiple images are to be added, a folder (path) with all the images can
% be added here
\graphicspath{ {figures/} }

% Separates the first part of the report/thesis in Roman numerals
\frontmatter


\begin{document}
\selectlanguage{english}

\begin{titlepage}
	\clearpage\thispagestyle{empty}
	\centering
	\vspace{1.2cm}

	% Titles
	{\large \textbf{Midterm Report} \par}
	\vspace{3cm}
	{\Huge \textbf{IPSTERS}} \\
  \vspace{1cm}
  {\Huge IPSentinel Terrestrial Enhanced Recognition System} \\
	\vspace{1cm}
	{\large \textbf{João Fonseca} \par}
	\vspace{.75cm}
	{\large \textbf{Advisor:} Prof. Fernando Lucas Bação \par}

	\vspace{1.3cm}
    \includegraphics[scale=0.2]{ims_logo.png}
  \vspace{1.3cm}

	{\normalsize NOVA Information Management School \\
		Instituto Superior de Estatística e Gestão de Informação \\
		Universidade Nova de Lisboa \par}

  \vspace{1.5cm}

	% Set the date
	{\normalsize \textbf \today \par}
	\pagebreak
\end{titlepage}

\chapter*{Abstract}

This report documents the work developed towards the research project "IPSTERS
- IPSentinel Terrestrial Enhanced Recognition System". It focuses on the
exploration of several machine learning (ML) techniques, covering different
stages of a Land Use/Land Cover Classification (LULC) pipeline. These
techniques aim to minimise problems typically found in this kind of data,
namely data ingestion, feature selection, data filtering and classification.



\tableofcontents{}

\clearpage

\listoffigures

\clearpage

\listoftables

\mainmatter

\chapter{Introduction}

The Copernicus programme is the European Union (EU) Earth Observation (EO)
programme, headed by the European Space Agency, and the developer of the
Sentinels EO satellites. The IPSentinel is the Portuguese infrastructure
developed by Direção Geral do Território (DGT) and Instituto Português do Mar e
da Atmosfera (IPMA) for storing and providing images of Sentinel satellites,
covering the Portuguese territory and its search and rescue area. This free EO
data has been used to inform environmental models, business strategies and
political decisions. However, this ever growing volume of data requires big
data workload that is overwhelming for Public Administration (PA) agencies. As
a result, the use of IPSentinel data has not been widely adopted by the PA that
would profit from it.

Often what these agencies need for their goals are digested data in the form of
specific class maps. These value-added products are often called level-3
products and are fundamental for land-management and for the country's
international commitments such as the estimation of CO2 emissions. These
products are mainly obtained by visual interpretation of high resolution
satellite imagery, requiring significant allocation of human resources from the
PA and taking a long time to produce, being one of the reasons for the low
update rate and low resolution. In the case of COS (Portuguese Land cover-land
use maps) they are produced every 5 years with 1ha resolution and EU CORINE
maps at least every 6 years with 25ha.

\section{Purpose and Objectives}

The main goal of this project is to explore the applications and limitations of
artificial intelligence (AI) algorithms with accelerated processing hardware
capabilities, as a unit of the IPSentinel for the digestion of large volumes of
remotely sensed data, to produce level-3 products for land applications with
the least amount of human intervention. We propose exploring two artificial
intelligence approaches, one applying active learning techniques and another
based on fuzzy logic.
\\
Data used
\\
Code availability here
\\
Research grant info

\section{Document Structure}


\chapter{Literature Review}

The state of the art on the main challenges identified for the project is shown
here. The multispectral imagery used in this project is targeted for a large
area (continental Portugal) and contains complex and highly correlated spectral
information. Although the presence of multicollinearity doesn't pose a problem
for the adequacy of modern ML models to predict a target variable
\cite{Farrell2019}, high dimensional data is difficult to process and therefore
strains the capacity of producing accurate LULC maps. Dimensionality reduction
techniques are used to address such an issue. These techniques allow the
selection of the most important feature within the image composites used,
allowing for 1) a clearer understanding of the most important features for LULC
classification, 2) accelerated model training and 3) avert the curse of
dimensionality \cite{Ghojogh2019}. Dimensionality reduction techniques can be
further subdivided into 2 parts, feature selection and feature extraction, both
techniques explored below.

The dataset is

\section{Feature Selection Methods}
ashdasf ahsd bjahsdb jahs.

\section{Feature Extraction Methods}
adasdasdasdas asdasdasda

\section{Data Filtering Methods}
% In order to add an unnumbered section/chapter the following code needs to be
% used
%\addcontentsline{toc}{section}{Uppgift 1}
\subsection*{Cluster-based methods}
asda asd ashd iuahsd iuahsd i.

\subsection*{Classifier-based methods}
Mitt data som ni kan se på sida, har jag använt för att göra den snygga figur
som finns på sida.


% Begins a figure float
\begin{figure}[H]
    \centering
    <<echo=FALSE, fig.height = 4, fig.width = 7>>=
    with(data=iris,plot(Sepal.Length,Sepal.Width,main="Min figur",col="Red", las = 1))
    @
    \caption{En figur}
    \label{fig:En-figur}
\end{figure}

\section{Classification Methods}
bla bala asdashd aisu hoaj.

\chapter{Methodology}

\chapter{Results and Discussion}



% Adding a bibliography if citations are used in the report
\bibliographystyle{plain}
\bibliography{references.bib}
% Adds reference to the Bibliography in the ToC
\addcontentsline{toc}{chapter}{\bibname}


\end{document}
